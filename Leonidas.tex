\documentclass[10pt,a4paper]{article}
\usepackage[utf8]{inputenc}
\usepackage[utf8]{inputenc}
\usepackage[T1]{fontenc}
\usepackage[french]{babel}
\usepackage{fancyhdr}
\pagestyle{fancy}

\renewcommand{\headrulewidth}{1pt}
\fancyhead[C]{\textbf{page \thepage}} 
\fancyhead[R]{Léonidas Ier de Sparte}

\renewcommand{\footrulewidth}{1pt}
\fancyfoot[C]{\textbf{page \thepage}} 


\title{Leonidas Ier de Sparte}
\author{yaniss pellier }
\date{Avril 2022}

\begin{document}

\maketitle
\newpage
\tableofcontents
\newpage

\section{Introduction}
Le sujet sera porté sur le 17e roi de Sparte, l'être qui a régné de 489 à 480 av. J-C, l'homme entré dans l'histoire, Léonidas Ier. De sa jeunesse à sa mort jusqu'au mythe...

   
\section{Modèle sociétal de la ville de Sparte}
\subsection{Royauté}
A cet époque, la Grèce n'était pas unifié et était formé de différente cités-etats. Les plus connu étant Sparte et Athène.
Contrairement à Athène, Sparte était une monarchie.


À partir de la réforme de Lycurgue au VIIe siècle av. J.-C., Sparte possède deux rois représentant l'un la famille des Agiades, l’autre celle des Eurypontides. Ils exercent conjointement des pouvoirs essentiellement militaires et religieux. Ce qui fait de Sparte une double monarchie.
\newline
\subsection{Mode de vie}
Dès l'âge de 7ans, les enfant spartiate sont contraint à quitter leur famille pour suivre un entrainement rigoureux où ils apprennent à ce battre mais également à endurer toutes les souffrances et privations. Seul le 1er fils du roi peut échapper à ce traitement. Celui là sera former à diriger et gouverner.
Cette éducation spartiate est baptisé "l'Agogé".
\newline

Grossièrement, voici comment tourne Sparte.
Ici, bien que respecté, le rôle des femmes dans cette socitété est d'engendrer de nouveaux combattants. Les esclaves, nommé les Hilotes, opprimé par les Spartiates, travaillent dans les champs pour les nourrir. Et cela se terminait en massacre dès la moindre erreur de leur part.
Et enfin les hommes de Sparte sont dévoué corps et âme à leur devoirs. Après un dur entrainement depuis leur 7ans, à 18ans, ils intègre leur première unité. Certaint pourront quitter l'armée dès 30ans pour fonder une famille et ainsi engendrer la prochaine génération de combattant.


Malgré cette socété totalement inhumaine qui plonge les spartiates dans un climat de violence toute leur vie, le modèle spartiate est admiré dans toute la Grèce car forcé de constater l'efficacité de leur modèle.



\section{Jeunesse et règne de Leonidas}
    Née vers 540 av. J-C, Léonidas est membre de la dynastie des Agiades, l'une des deux familles royale. Il est le troisième fils du roi Anaxandridas II ; ses frères sont Dorieus, son aîné, et Cléombrote, son cadet. Cléomène Ier, le plus âgé, est quant à lui son demi-frère. Léonidas n'est pas destiné à être le roi car n'étant pas l'ainé de la famille. Il aura donc une enfance commune à tout les Spartiate en suivant l'Agogé.
    
    A ses 20ans, Léonidas termine sa formation. C'est officiellement un guerrier spartiate. C'est à ce moment qu'il apprendra la mort de son père, désormais remplacer par Cléomène Ier, comme dit plus haut: son demi-frère ainé. Quelque années plus tard, Léonidas épousera Gorgo, la fille de ce dernier.
    
    La situation a Sparte devient critique face à la monté des Perses. Cléomène Ier tentera de renverse un roi voisin qui comptait les aider, pour ralentir leur progression mais échouera et démissionera. Il sera arrêté et jeter en prison, où il se suicidera. N'ayant pas de descendants, Léonidas prend la place de son frère ainé ainsi que du pouvoir. Ainsi débute son règne. Malheuresement, l'un des plus grand empire de l'humanité, les Perses, convoite toujours Sparte...
    
\section{La bataille des Thermopyles}
\subsection{avant la bataille}
  L'une des bataille les plus célébre de la Grèce antique. Après que l'armée de Xerses Ier, le roi perse, débarque sur une crique nommé Thermopyle. L'armée de Xerxes Ier écrasant numériquement\footnote{60 000 par rapport à plus d'1 million} celle de Léonidas, celui-ci à l'idée d'une attaque suicide qui en cas de victoire infligera une lourde défaite au Perses. Léonidas choisit de combattre les Perses avec 300 hommes. Ce qui est peu, mais plus facilement gérable. A cela venait s'ajouter 4 000 à 700 000 autres soldats d'autre cités-etats.
  
  \subsection{la bataille perdu mais la guerre gagné}
  Arrivé sur la crique, commence la bataille et les spartiates arriveront à repousser les Perses toute une journée. Le lendemain, un traitre grec donne des informations à Xerses pour prendre en tenaille les spartiates. A la fin de la deuxième journée de combat, 10 000 hommes de Xerses contourne la crique et prenne les spatiates par l'arrière. Au troisième jour, les Perses prennent le dessus et Léonidas est tué et son armée décimée.
  
  Grâce à ce sacrifice, les grecs prirent concience de la menace Perse. Les cités-etats s'alliront contre Xerxes et provoqueront sa chute.

\section{Légende vivante}
    Les sources de cette histoire proviendront de Hérodote, considéré comme l'un des premier historien de l'histoire.
    Plusieurs oeuvres traite de cette histoire:
    \begin{itemize}
        \item le livre 300 de Frank Miller et son adaptation au cinéma de Zack snyder
        \item Jacques-Louis David peint Léonidas aux Thermopyles (1814)
        \item le film américain  La Bataille des Thermopyles de Rudolph Maté (1962)
    \end{itemize}
    
\section{Mathématique de la Grèce antique}
\subsection{Pythagore}
Pythagore est un mathématicien reconnu de la grèce antique. Il a découvert entre autre un moyen de calculer tous les triangles rectangles.
"Dans un triangle rectangle, le carré de la longueur de l’hypoténuse est égal à la somme des carrés des longueurs des deux autres côtés." Voici la formule: a² = b² + c².
\subsection{Ptolémée}
Un autre grand mathématicien de la grèce antique. Voici son théorème: "Un quadrilatère convexe est inscriptible si et seulement si le produit des longueurs des diagonales est égal à la somme des produits des longueurs des côtés opposés." Il peut être traduit par: ${\displaystyle AC\cdot BD=AB\cdot CD+BC\cdot AD.}$

\end{document}
